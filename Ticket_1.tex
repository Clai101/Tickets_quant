\section{Билет}
\subsection{соотношение неопределенности}
Говрит o том, что неовзмжно точно померить и координату и импульс.
 Так как если велечины измеримы то $\exists$ такая есть общий базис 
 из собствнных функций. Тогда если велечинам соответствуют 
 операторы $\hat A,\ \hat B$ то:
\begin{equation*} 
 \hat A \hat B \ket{n} = \hat B \hat A  \ket{n} \ \implies\ 
 \inner{\hat B \hat A - \hat A \hat B} \ket{n} = 
 \insqr{\hat A, \hat B}\ket{n} = 0,
\end{equation*}
а следовательно 
\begin{equation} 
 \insqr{\hat A, \hat B} = \hat 0.
\end{equation} 
\\
Например, для импульса и координаты ссотношение неопредленности 
выглядит следущим образом:
\begin{equation} 
 \Delta x \cdot \Delta p \geq \cfrac{\hslash}{2}.
\end{equation} 

\subsection{Волновая функция}
Это удовлетворяет уравнению:
\begin{equation} 
 p (x, t) = \abs{\psi (x, t)}^2.
\end{equation} 
Для нее выполнятся условие нормировки 
\begin{equation} 
 \braket{\psi (x)}{\psi (x)} = \int_{-\infty}^{\infty} 
  \abs{\psi (x)}^2 dx = 1.
\end{equation} 
   
\subsection{Принцип суперпозиции}
Пусть состояне частици описыватся волновой функций $\psi_1$
 и втоже время $\psi_2$ приэтом каждой функции соответственно удовлетворят
 состояния $A_1, \ A_2$ то также эта частица будет описыватся волновой 
 волновой функцией $\psi_3 = c_1 \psi_1 + c_2 \psi_2$ где $c_1, \ c_2 \in 
 \com$ c соответствующим состоянем $A_3$ это сотояние при 
 измерении может принять значения $A_3 = A_1 \lor A_3 = A_2$.
 









