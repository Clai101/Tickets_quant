\section{Билет}
\subsection{соотношение неопределенности}
Говрит о том, что неовзмжно точно померить и координату и импульс.
 Так как если велечины измеримы то $\exists$ такая есть общий базис 
 из собствнных функций. Тогда если велечинам соответствуют 
 операторы $\hat A,\ \hat B$ то:
$$ \hat A \hat B \ket{n} = \hat B \hat A  \ket{n}$$
А следовательно 
$$\insqr{A, B} = \hat 0$$
\\
Точность задается следующим соотношением. 
$$\Delta x \cdot \Delta p \geq \cfrac{\hslash}{2}$$

\subsection{Волновая функция}
Это удовлетворяет уравнению:
$$
p (x, t) = \abs{\psi (x, t)}^2
$$
Для нее выполнятся условие нормировки 
$$\braket{\psi (x)}{\psi (x)} = \int_{-\infty}^{\infty}
 \abs{\psi (x)}^2 dx = 1$$


\subsection{Принцип суперпозиции}
Пусть состояне частици описыватся волновой функций $\psi_1$
 и втоже время $\psi_2$ приэтом каждой функции соответственно удовлетворят
 состояния $A_1, \ A_2$ то также эта частица будет описыватся волновой 
 волновой функцией $\psi_3 = c_1 \psi_1 + c_2 \psi_2$ где $c_1, \ c_2 \in 
 \com$ с соответствующим состоянем $A_3$.









