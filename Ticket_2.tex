\section{Билет}
\subsection{Операторы}
\subsubsection{Определение}
Оператор должен удовлетворять основному свойству
\begin{equation} 
 \hat A f(x) = \xi (x), 
\end{equation} 
То-есть оператор переводит Функцию в функцию, для 
любых двух операторов выполняется свойство асоциативности, 
но не для любых операторов выполнятся свойство комуативности. 
Как видно из \ref{eq:1}{уравнениея 1} из комутативности следует 
одновремнная измеримость.
\subsubsection{Основное свойство}
Оператор линеен
\begin{equation} 
  \hat A \inner{c_1 \ket{\psi_1} + c_2 \ket{\psi_2}} = 
  c_1 \hat A \ket{\psi_1} + c_2 \hat A \ket{\psi_2}
\end{equation} 
\subsubsection{Транспонирование}
\begin{equation} 
 \bra{\psi} \hat A \ket{\varphi} = \bra{\varphi} \hat A^T \ket{\psi} 
\end{equation}

\subsection{Связь измеримых велечин и операторов}
Во-первых, $\psi_n$ (обозначаем как $n$) собственная функция 
$\hat A $ если выполняется:
\begin{equation} 
 \hat A \ket{\psi_n} = A_n \ket{\psi_n}
\end{equation} 
При этом $A_n \in \com$ это величина, но что бы она была измерима 
надо чтобы  $A_n \in \re$, то условие выполняется в случае если: 
\begin{equation} 
    \hat A^\dag : = \hat A^{T *}.
\end{equation} 

\begin{equation} 
    \hat A^\dag \leqslant \hat A.
\end{equation} 










