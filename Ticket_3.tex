\section{Билет}
\subsection{Различные пространства и представления}
\tab До этого мы работали только с одни представлением неакцентируя 
на этом внимания теперь надо определить что это.

Как мы помним для собственных функций оператора $\hat A$ обязательно 
выполняется услловие \ref{eq:1.6}{нормировки}, пока запомим 
это и вспомним еще одно свойтво что собственные вектора образуют 
базис тогда:
\begin{equation*} 
 \forall \ket{\psi} \per \ket{\psi} = \sum a_i \ket{\phi_i} \implies 
 \braket{\phi_j}{\psi} = \sum a_i \delta_{ij} = a_i.
\end{equation*}  

Совокупность велечин $a_i = \braket{\phi_i}{\psi}$ называтся
волновой функцией в $A$-представелнии. 

Предлагаю рассмотреть примерпр с $\hat x, \ \hat p$. 
\begin{equation*} 
\hat p \ket{p} = \hat p \hat 1_x \ket{p} = 
\hat p \ket{x'} \braket{x'}{p} = 
\int_{\re} dx' \hat p \ket{x'}\braket{x'}{p} = 
\int_{\re} dx' \hat p \ket{x'} \phi_p(x'),
\end{equation*}
в то же время  
\begin{equation*} 
\hat p \ket{p} = p \ket{p} = \hat 1 p \ket{p} = 
\int_{\re} dx' p \ket{x'} \braket{x'}{p} = 
p \int_{\re} dx' \ket{x'} \phi_p(x').
\end{equation*} 

Домножим на $\bra{x}$:
\begin{equation*} 
\int_{\re}dx'\bra{x} \hat p \ket{x'} \phi_p(x') = 
\int_{\re}dx' p \braket{x}{x'} \phi_p(x') =
\int_{\re}dx' p \delta(x - x') \phi_p(x') = 
p \phi_p(x) = \hat p \phi_p.
\end{equation*} 

И получим:
\begin{equation} 
\bra{x} \hat p \ket{x'} = \hat p \delta(x - x') = 
- i \hslash \difh{}{x} \delta(x - x').
\end{equation}

Теперь давайте рассмотрим переход операторов от любого 
a-пердставления к b-представлению, пусть для 
$\hat A \ket{a} = a \ket{a}$ и аналогичо для $\hat B$:
\begin{equation*} 
  \bra{\psi(a)} = \sum a_i \bra{a_i(a)}, \ \ 
  \hat A \ket{\psi(a)} = \sum b_i \ket{a_i(a)}.
\end{equation*} 

\begin{equation*} 
    \bra{a_j(a)} \hat A \ket{\psi(a)} = 
    \sum_{i = 0} \bra{a_j(a)} b_i \ket{a_i(a)} = 
    b_i \bra{a_i(a)} \ket{a_i(a)} = b_i,
\end{equation*} 
с другой стороны 
\begin{equation*} 
    \bra{a_j(a)} \hat A \ket{\psi(a)} = 
    \bra{a_j(a)} \hat A \sum_{i = 0} a_i \ket{a_i(a)} = 
    \sum_{i = 0} \bra{a_j(a)} A_i a_i \ket{a_i(a)} = 
    \sum_{i = 0} A_i a_i.
\end{equation*} 
То есть 
\begin{equation} 
 b_i = \sum_{i = 0} A_i a_i
\end{equation} 

Таким образом полкчим что оператор $\hat A$ можно 
выразит через собственные функции.