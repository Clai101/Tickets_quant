\documentclass{article}
%Кодировка и язык
\usepackage[T2A]{fontenc}
\usepackage[utf8]{inputenc}
\usepackage[russian]{babel}

%Формат документа
\usepackage[a4paper, total={6.5in, 10in}]{geometry}

\usepackage{subfiles}
\usepackage{amsmath}
\usepackage{physics}
\usepackage{amssymb}

%Добавляет коробочки, можно складывать туда текст)
\usepackage{collectbox}

%Картинки
\usepackage{graphicx, epstopdf}
\usepackage{float}

%Графики
\usepackage{pgfplots}
    \usetikzlibrary{
        angles,
        quotes,
    }
%Подфигуры
\usepackage{subfigure}

%Сокращения, упрощения
\usepackage{cancel}

%Ссылки
\usepackage[unicode, pdftex]{hyperref}

%Улучшеные шрифты
\usepackage{bm}
\usepackage{titleps}

%Cтруктуры
\usepackage{array}

%Сокращения
\newcommand{\inner}[1]{\left( #1 \right)}
\newcommand{\infig}[1]{\left{ #1 \right}}
\newcommand{\insqr}[1]{\left[ #1 \right]}

\newcommand{\com}{\mathbb{C}}
\newcommand{\re}{\mathbb{R}}
\newcommand{\nat}{\mathbb{N}}

\makeatletter
\newcommand{\sqbox}{%
    \collectbox{%
        \@tempdima=\dimexpr\width-\totalheight\relax
        \ifdim\@tempdima<\z@
            \fbox{\hbox{\hspace{-.5\@tempdima}\BOXCONTENT\hspace{-.5\@tempdima}}}%
        \else
            \ht\collectedbox=\dimexpr\ht\collectedbox+.5\@tempdima\relax
            \dp\collectedbox=\dimexpr\dp\collectedbox+.5\@tempdima\relax
            \fbox{\BOXCONTENT}%
        \fi
    }%
}
\makeatother
\newcommand{\mergelines}[2]{
\begin{tabular}{llp{.5\textwidth}}
#1 \\ #2
\end{tabular}
}
\newcommand\tab[1][0.51cm]{\hspace*{#1}}
\newcommand\difh[2]{\frac{\partial #1}{\partial #2}}



\title{Билеты\\ По квантовой механике}
\author{Автор:\\
Matvei Karibdzhanov}
\date{January 2023}

\newpagestyle{main}{
\setfootrule{0.4pt}
\setfoot{}{\thepage}{Билет №\thesection}}
\pagestyle{main}


\begin{document}
\pagestyle{main}

\maketitle
\tableofcontents

\section{Билет}
\subsection{соотношение неопределенности}
Говрит o том, что неовзмжно точно померить и координату и импульс.
 Так как если велечины измеримы то $\exists$ такая есть общий базис 
 из собствнных функций. Тогда если велечинам соответствуют 
 операторы $\hat A,\ \hat B$ то:
\begin{equation*} 
 \hat A \hat B \ket{n} = \hat B \hat A  \ket{n} \ \implies\ 
 \inner{\hat B \hat A - \hat A \hat B} \ket{n} = 
 \insqr{\hat A, \hat B}\ket{n} = 0,
\end{equation*}
а следовательно 
\begin{equation} \label{eq:1}
 \insqr{\hat A, \hat B} = \hat 0.
\end{equation} 
Согласн принципу неопределенности
\begin{equation} 
 \Delta_{\psi} A \Delta_{\psi} B \  
\end{equation} 
Например, для импульса и координаты ссотношение неопредленности 
выглядит следущим образом:
\begin{equation} 
 \Delta x \cdot \Delta p \geq \cfrac{\hslash}{2}.
\end{equation} 
Для доказательства этого соотношения представим, что
\begin{equation} 
 \insqr{\hat f, \hat g} = - i \hslash \hat C,
\end{equation}  
Теперь давайте в ведем:
\begin{equation} 
  \sigma \hat f = \sqrt{\ave{\inner{\hat f - \ave{\hat f}}^2 }} = \sqrt{\ave{\inner{ \ave{\hat f_0}^2 }}},
\end{equation} 
\begin{equation} 
 \ket{\phi } = \inner{\alpha \hat f + i \hat g }\ket{\psi}.
\end{equation} 
Так же не забываем требование эрмитовости операторов тогда:
\begin{equation} 
 \braket{\phi}{\phi} = \bra{\psi} \inner{\alpha^2 \hat f_0^2 + \hat g_0^2 + i\alpha \insqr{\hat f_0, \hat g_0}} \ket{\psi} = 
 \alpha^2 \ave{\hat f_0^2} +  \ave{ \hat g_0^2} + \alpha \hslash \ave{\hat C}
\end{equation} 

\subsection{Волновая функция}
Это удовлетворяет уравнению:
\begin{equation} 
 p (x, t) = \abs{\psi (x, t)}^2.
\end{equation} 
Для нее выполнятся условие нормировки 
\begin{equation} 
 \braket{\psi (x)}{\psi (x)} = \int_{-\infty}^{\infty} 
  \abs{\psi (x)}^2 dx = 1.
\end{equation} 
   
\subsection{Принцип суперпозиции}
Пусть состояне частици описыватся волновой функций $\psi_1$
 и втоже время $\psi_2$ приэтом каждой функции соответственно удовлетворят
 состояния $A_1, \ A_2$ то также эта частица будет описыватся волновой 
 волновой функцией $\psi = c_1 \psi_1 + c_2 \psi_2$ где $c_1, \ c_2 \in 
 \com$ c соответствующим состоянем $A$ это сотояние при 
 измерении может принять значения $A = A_1 \lor A = A_2$.

Пример частица может находитсяв точке $X_1$ этому полодение соответствует 
 собственная функция $\psi_1$, но также она модет быть в точке $X_2 - \psi_2$ 
 Тогда если мыбудем пытаться измерить полодение чатици то мыбудем получать
 значения $x_1 \lor x_2$, приэтом вероятность: 
 \begin{equation} 
  \mathbb{P}(X = x_1) = \braket{\psi}{\psi_1}.
 \end{equation}

 













\end{document}